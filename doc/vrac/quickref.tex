\documentclass[a4paper]{letter}
\usepackage[landscape,vcentering,dvips]{geometry}
\geometry{papersize={210mm,297mm},total={260mm,180mm}}
\usepackage{lscape}
\usepackage{multicol}
\usepackage{fontspec}
\usepackage{amssymb}
\newcommand{\section}[1]{
\rule{0.5cm}{0.1pt} \textbf{\large {#1}} \hrulefill
}
\newcommand{\subsection}[2]{
\rule{0.5cm}{0.1pt} {\scriptsize [{#1}]} {#2}
}
\newcommand{\collorg}{\texttt{\textit{collorg}}}
\newcommand{\cog}{\texttt{cog}}
\newcommand{\API}{\texttt{API}}
\newcommand{\attributs}{\texttt{attributs}}
\newcommand{\relations}{\texttt{relations}}
\newcommand{\controller}{\texttt{controller}}
\newcommand{\templates}{\texttt{templates}}
\newcommand{\argument}{\texttt{argument}}
\newcommand{\ctrlarg}{\texttt{controller argument}}
\pagenumbering{gobble}
\columnsep 1cm





\begin{document}
\begin{multicols}{3}


\begin{center}
{\Large Manuel de programmation \collorg}
\end{center}





\section{La commande \texttt{cog}}

Permet de gérer une application \collorg\ (initialisation, mises à jour, ...).

\subsection{\cog}{Créer une nouvelle application \collorg}

\begin{description}
\item[\texttt{cog init}] Initialise une application. Crée un répertoire correspondant au dépôt \collorg\ de l'application.
\begin{scriptsize}
\begin{verbatim}
# cog init -d my_app
[...]
# tree -a my_app/
my_app/
├── .cog
│   └── config
├── collorg_app
│   ├── __init__.py
│   └── my_app
│       ├── db
│       │   └── __init__.py
│       └── __init__.py
├── __init__.py
├── Makefile
├── scripts
├── setup.py
└── sql
\end{verbatim}
\end{scriptsize}
\end{description}
\subsection{\cog}{Dans un dépôt \collorg}
\begin{description}
\item[\texttt{cog make}] Propage les modifications apportées à l'application.
\item[\texttt{cog struct}] Affiche la structure de la base de donnée de l'application, d'un schéma (-s) ou d'une table (\texttt{-t}).

\begin{scriptsize}
\begin{verbatim}
# cog struct
# cog struct -s collorg.core
# cog struct -t collorg.core.base_table
\end{verbatim}
\end{scriptsize}
\item[\texttt{cog graph}] Affiche le graphe de la base de donnée ou d'un schéma de la base de donnée (\texttt{-s}).

\begin{scriptsize}
\begin{verbatim}
# cog graph
# cog graph -s collorg.access
\end{verbatim}
\end{scriptsize}
\end{description}





\section{Création d'un type de données}

La création d'un nouveau type de données passe par la création d'une nouvelle table dans la base de données.

\begin{scriptsize}
\begin{verbatim}
CREATE SCHEMA new_schema;
CREATE TABLE new_schema.new_table (
   attr1 TEXT PRIMARY KEY,
   attr2 int
);
\end{verbatim}
\end{scriptsize}

Une fois la table créée, la commande \texttt{cog make}, génère le code nécessaire à son utilisation avec l'API \collorg.

\begin{scriptsize}
\begin{verbatim}
# cog make
[...]
# tree collorg_app/my_app/db/
collorg_app/my_app/db/
├── __init__.py
└── new_schema
    ├── __init__.py
    └── new_table
        ├── cog
        │   └── __init__.py
        ├── templates
        │   └── __init__.py
        └── __init__.py
\end{verbatim}
\end{scriptsize}

Un nouveau module est créé par la commande \texttt{cog make} dans le répertoire~:\\
\texttt{collorg\_app/my\_app/db/new\_schema/new\_table}. Le fichier \texttt{\_\_init\_\_.py} contient la définition de la classe \texttt{New\_table}. Le FQTN (\textit{fully qualified table name}) de cette table est la chaîne \texttt{'new\_schema.new\_table'} (concaténation du nom du schéma et du nom de la table, cf. \texttt{CREATE TABLE}).



\section{L'API de \collorg}

\texttt{Controller.Model.Relation.Attribute}

\subsection{\API}{Le contrôleur}

Il permet entre autres d'initialiser la connexion à la base de données. Les lignes suivantes seront toujours présentes au début d'un script \collorg.

\begin{scriptsize}
\begin{verbatim}
#!/usr/bin/env python
#-*- coding: utf-8 -*-

from collorg.controller.controller import Controller
model = Controller().model
relation  = model.relation
\end{verbatim}
\end{scriptsize}

L'instanciation d'un contrôleur sans argument (\texttt{Controller()}) sous entend que l'exécution du script est toujours faite à partir d'un dépôt \collorg. \texttt{Controller('my\_app')} en revanche, établit explicitement une connexion sur la base de donnée/application \texttt{my\_app}.


\subsection{API}{Les relations}

Une relation est le type d'objet de base permettant de manipuler une table dans l'API \collorg. Pour instancier un objet de ce type, on utilise la méthode~:\\ \texttt{relation = Controller().model.relation}\\ (cf. script de base contrôleur).

\begin{scriptsize}
\begin{verbatim}
nt1 = relation('new_schema.new_table')
nt2 = relation('new_schema.new_table')
\end{verbatim}
\end{scriptsize}

Cinq méthodes sont disponibles pour interagir avec la base de données~:

\begin{description}
\item[\texttt{Relation.select()}] Déclenche l'extraction des données de la base. Une relation est itérable. L'invocation de \texttt{select} est implicite dans un contexte de liste~:

\begin{scriptsize}
\begin{verbatim}
for elt in relation.select():
\end{verbatim}
\end{scriptsize}
s'écrit
\begin{scriptsize}
\begin{verbatim}
for elt in relation:
\end{verbatim}
\end{scriptsize}
\item[\texttt{Relation.insert()}] Insert la donnée dans la base.
\item[\texttt{Relation.update(nval)}] Met à jour la donnée dans la base avec \texttt{nval}.
\item[\texttt{Relation.delete()}] Détruit l'ensemble référencé de la base.
\item[\texttt{Relation.is\_empty()}] Retourne \texttt{True} si la contrainte posée sur l'objet \texttt{relation} definit un ensemble vide.

\begin{scriptsize}
\begin{verbatim}
obj = relation(Type)
obj.attr_.value = 'a value'
if obj.is_empty():
    obj.insert()
\end{verbatim}
\end{scriptsize}
\end{description}



\subsection{\API:\relations}{Les opérateurs algébriques}

Soient \texttt{a1} et \texttt{a2} deux relations d'un même type \texttt{FQTN}.

\begin{scriptsize}
\begin{verbatim}
a1 = relation(FQTN)
a2 = relation(FQTN)
\end{verbatim}
\end{scriptsize}
Notons $A_{1}$ et $A_{2}$ les ensembles correspondants. Alors~:

\begin{center}
\begin{scriptsize}
\begin{tabular}{|c|c|}
\hline
Python & Set \\
\hline
$a1\ + a2$ & $A_{1} \cup A_{2}$ \\
$a1\ *\ a2$ & $A_{1} \cap A_{2}$ \\
$a1\ - a2$ & $A_{1}\ \backslash\ A_{2}$ \\
$-a1$ & $A_{1}^\complement$ \\
$a1\ in\ a2$ & $A_{1} \subset A_{2}$ \\
$a1\ ==\ a2$ & $A_{1} = A_{2}$ \\
\hline
\end{tabular}
\end{scriptsize}
\end{center}

\subsection{\API}{Les attributs}

Les attributs d'une relation correpondent aux attributs de la table. Ils portent les mêmes noms suffixés par \_. Reprenant notre relation de type \texttt{new\_schema.new\_table}, si \texttt{nt1~=~relation('new\_schema.new\_table')}, alors les attributs \texttt{nt1.attr1\_} et \texttt{nt1.attr2\_} représentent les champs correspondants.

\subsection{\API:\attributs}{La propriété \texttt{value}}

La propriété \texttt{value} d'un attribut permet de récupérer sa valeur ou de la positionner.

Récupération d'une valeur (exemples) :

\begin{scriptsize}
\begin{verbatim}
valeur = nt1.attr2_.value
if nt1.attr1_.value == 'une valeur':
    # faire quelque chose
\end{verbatim}
\end{scriptsize}

Positionnement d'une valeur (exemples) :

\begin{scriptsize}
\begin{verbatim}
nt1.attr1_.value = 'abc'
nt2.attr1_.value = ('xyz', 'tuv'), 'in'
\end{verbatim}
\end{scriptsize}

Le postionnement d'une valeur peut être accompagné d'un opérateur de comparaison \texttt{SQL}. Les opérateurs reconnus sont :

\begin{scriptsize}
'=', '!=', '>', '>=', '<', '<=', 'is', 'is not', 'like', 'not like',
'ilike', 'not ilike', '~', '~*', '!~', '!~*', 'in', 'not in'
\end{scriptsize}

\section{Les templates}

Les templates sont des fichiers suffixés \texttt{.cog} et utilisant le langage de templates de \collorg. Ces templates sont systématiquement positionnées dans le sous répertoire \texttt{templates} du module correpondant.

\begin{scriptsize}
\begin{verbatim}
collorg_app/my_app/db/
└── new_schema
    └── new_table
        └── templates
            └── __init__.py
\end{verbatim}
\end{scriptsize}


Les templates utilisent un langage de templates propre à \collorg. Ce langage permet de conserver la richesse de Python tout y en ajoutant des zones d'affichage.

\subsection{\templates}{Les balises du langage}

Le caractère $_{\smallsmile}$ représente un espace.
\begin{description}
\item[\texttt{\#$_{\smallsmile}$>>>}] Entrée dans du code Python.
\item[\texttt{\#$_{\smallsmile}$---}] Zone d'affichage et marque d'indentation hors code Python. Le code qui suit sera affiché tel quel.
\item[\texttt{\#$_{\smallsmile}$+++ <variable>}] Zone d'affichage. Le code qui suit (jusqu'à la balise \texttt{\#$_{\smallsmile}$--- <variable>} est stocké tel quel (pour affichage) dans la variable \texttt{<variable>}
\item[\texttt{\{\% code\_python \%\}}] Code embarqué dans une zone d'affichage.
\end{description}

\subsection{\templates}{Les variables}

Un certain nombre de variables sont accessibles au niveau d'une template. Les variables notées \ctrlarg\ sont mises en place par le \controller.

\begin{description}
\item[self] \argument\\ l'objet d'appel (objet de type \texttt{Relation}),
\item[cog\_charset] \ctrlarg\\ le charset à utiliser pour la template,
\item[cog\_user] \ctrlarg\\ l'utilisateur ayant appelé la template,
\item[cog\_environment] \ctrlarg\\ l'environnement d'appel de la template (!= self),
\item[kwargs] \argument\\ le dictionnaire contenant les arguments \textit{clefvaleur} passés,
\item[\_] la fonction d'i18n,
\item[html] le module de génération de code HTML à partir des relations \collorg.
\end{description}

\subsection{\templates}{Exemple}


Appelons la template dont le script suit \texttt{w3list.cog}. Elle reçoit un argument nommé \texttt{title}.

\begin{scriptsize}
\texttt{new\_schema/new\_table/templates/w3list.cog}
\end{scriptsize}

\begin{scriptsize}
\begin{verbatim}
# >>>
"""
New_table.w3list(title='A title')
"""
title = kwargs['title']
# +++ output
# >>>
for elt in self:
    # ---
    <li> <em></em></li>
# --- output
# >>>
if output:
    # ---
    <h2></h2>
    <ul></ul>
\end{verbatim}
\end{scriptsize}

\texttt{self} est un objet de type \texttt{'new\_schema.new\_table'}. S'il n'est pas vide, la liste des objets est affichée au format \texttt{HTML}.

La commande \texttt{cog make} génère pour chaque fichier \texttt{.cog} un fichier \texttt{.py}. Chaque fichier template correspond à une nouvelle méthode de la classe correspondant au type. Le nom de la méthode est celui du fichier sans l'extension \texttt{.cog}.

Exemple d'appel de cette template~:
\begin{scriptsize}
\begin{verbatim}
nt = relation('new_schema.new_table')
nt.w3list(title='The list')
\end{verbatim}
\end{scriptsize}


\hrulefill
\begin{scriptsize}
2014 Joël Maïzi\\
http://www.collorg.org/\\
GNU Free Document License, extend for your own use
\end{scriptsize}

\begin{scriptsize}
\begin{verbatim}
\end{verbatim}
\end{scriptsize}




\newpage
\end{multicols}
\end{document}
